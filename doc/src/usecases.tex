\section{Zastosowania RBM}
    \paragraph{}
        Sieci neuronowe wykorzystywane są do wielu zadań przy których konwencjonalne
        podejście jest mało praktyczne. Jednym z takich zastosowań jest rozpoznawanie
        odręcznego pisma i jego konwersja na dokument tekstowy. Ponadto algorytm taki
        potrafi dostosować się do użytkownika i z czasem rozpoznawać jego styl pisma
        efektywniej i dokładniej. Ma to zastosowanie w tablicach interaktywnych oraz
        smartfonach.

    \paragraph{}
        Najczęściej algorytmy sztucznej inteligencji wykorzystywane są w celu rozpoznania
        jakiegoś schematu w danych które otrzymują. Podczas treningu kontrolujemy dane wejściowe
        w celu nauczenia algorytmu rozpoznawania interesujących nas danych. Ma to szerokie zastosowanie,
        algorytm może rozpoznawać mowę w dźwięku, gatunki zwierząt w zdjęciach, choroby z listy objawów.
	\paragraph{}
	    Do podstawowych zastosowań ograniczonej maszyny Boltzmanna zaliczamy różne metody regresji takie
        jak regresja liniowa czy regresja logistyczna. Innym zastosowaniem tej sztucznej sieci neuronowej
        jest filtrowanie zespołowe, uczenie schematów czy modelowanie tematycze.