\section{Przykład użycia RBM}
    \paragraph{}
	Nasz projekt ograniczonej maszyny boltzmanna o koncepcje biblioteki książek. 
	Użytkownik podaje swoje preferencje w sposób binarny - książka podoba mu się lub nie. 
	Przed tym następuje proces trenowania maszyny oparty na preferencjach innych użytkowników biblioteki. 
	Wynik jest następie przepuszczny w dodatnią fazę \textit{kontrastowej rozbieżności},
	aby znaleść współne cechy pozycji które podał użytkownik (obliczenie ukrytych neuronów),
	następie przepuszczny w dodatnią fazę \textit{kontrastowej rozbieżności}, 
	aby znaleść inne pozycje które mogą spodobać się użytkownikowi.
    \paragraph{}
	Obracowana maszyna Boltzmanna zawiera 24 widoczne neurony oraz 4 ukryte, które oznaczają cechy wspólne cechy ulubionych książek, które wybrał 				użytkownik