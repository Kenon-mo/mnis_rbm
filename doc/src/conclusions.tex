\section{Wnioski}
    \paragraph{}
	Zaprezentowana ograniczona maszyna Boltzmanna jest w stanie klasyfikować różne dane za pomocą cech wspólnych.
	Pokazuje to a jaki sposób można \textit{nauczyć} komputer rozróżniać różne informacje na pomocą wspólnych danych.
	Podobne algorytmy są często stosowane na różnych portalach oferujących różne filmy, audiobooki czy gry w ramach miesięcznej subskrybcji.
	Używając ograniczonej maszyny Boltzmanna firma jest w stanie odtrzymać naszą uwagę przy ich produktach oraz skłonić nas
	do kolejnego zakupi comiesięcznej usługi.
    \paragraph{}
	Warto również wspomnieć o wykorzystaniu danych treningowych podczas korzystania z takich usług. Algorytm aplikacji,
	z której korzystamy przez dłuższy czas, dane dostaje w sposób ciągły, więc minimalizacja błędu między doborem odpowiedmich kategorii
	będzie o wiele lepsza niż w przedstawionym tutaj algorytmie. 
