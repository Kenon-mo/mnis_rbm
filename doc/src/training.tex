\section{Trenowanie sieci neuronowej}
    \paragraph{}
        Trenowanie sieci neuronowej jest elementem, który ma na celu dopasowanie wag do odpowiednich danych wejściowych.
        Obliczenia dokonują się na podstawie kontrastowej rozbieżności (ang. \textit{Constrastive divergance}), która ma na celu
        dopasowanie wag przez różnice rzeczywistego wyniku działania algorytmu z "wirtualnym" wynikiem wygenerowanym na
        podstawie danych wyuczonych przez algorytm. Całość jest ograniczana przez tzw. współczynnik uczenia, który mówi
        nam o ile algorytm powienien zmniejszyć różnicę otrzymaną z CD.
    \paragraph{}
        Operacje trenowania macierzy najwygodniej przedstawić na macierzach. Za pomocą operacji transpozycji czy mnożenia
        macierzowego można zaoszczdzić sporo czasu na opracowywanie algorytmu RBM. Do treningu będzie potrzebna funkcja sigmoid,
        która przyjmuje wartości od 0 do 1,  a zmiana pomiędzy nimi następuje dla argumentów bliskich zeru.