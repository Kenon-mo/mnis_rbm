\section{Testowanie sieci neuronowej}
    \paragraph{}
	Testowanie sieci neuronowej jest najłatwiejszym etapem algorytmu ograniczonej maszyny Bolzmanna. Najważniejsze jest, żeby było to przeprowadzone po  		\textit{wytrenowaniu} maszyny przy pomocy danych treningowych. Otrzymany wynik algorytmu jest najbardziej przewidywaną odpowiedzią, która zadowoli 			użytkownika. Testować maszynę Boltzmanna możemy szukając ukryte neurony przy określonych danych testowych lub na odwrót.
    \paragraph{}
	Testowanie jest podobne do początkowego etapu trenowania maszyny boltzmanna. Dane użytkownika w postaci macierzy mnożymy przez wytrenowaną 			macierz wag w celu szukania ukrytych neuronów, lub wytrenowaną, transponowaną macierz wag w celu szukania widocznych neuronów. Następie obliczamy 		zero jedynkowe prawdopodobieństwo na podstawie wyników mnożenia macierzy otrzymanych z funkcji sigmoid. Otrzymane dane są najbardziej zbliżone do 		preferencji użytkownika według zadanych przedtem danych treningowych.